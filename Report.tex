% VDE Template for EUSAR Papers
% Provided by Barbara Lang und Siegmar Lampe
% University of Bremen, January 2002
% English version by Jens Fischer
% German Aerospace Center (DLR), December 2005
% Additional modifications by Matthias Wei{\ss}
% FGAN, January 2009

%-----------------------------------------------------------------------------
% Type of publication
\documentclass[a4paper,10pt]{article}
%-----------------------------------------------------------------------------
% Other packets: Most packets may be downloaded from www.dante.de and
% "tcilatex.tex" can be found at (December 2005):
% http://www.mackichan.com/techtalk/v30/UsingFloat.htm
% Not all packets are necessarily needed:
\usepackage[T1]{fontenc}
\usepackage[latin1]{inputenc}
%\usepackage{ngerman} % in german language if required
\usepackage[nooneline,bf]{caption} % Figure descriptions from left margin
\usepackage{times}
\usepackage{multicol}
\usepackage{amsmath}
\usepackage{amssymb}
\usepackage[dvips]{graphicx}
\usepackage{epsfig}
\input{tcilatex}
%-----------------------------------------------------------------------------
% Page Setup
\textheight24cm \textwidth17cm \columnsep6mm
\oddsidemargin-5mm                 % depending on print drivers!
\evensidemargin-5mm                % required margin size: 2cm
\headheight0cm \headsep0cm \topmargin0cm \parindent0cm
\pagestyle{empty}                  % delete footer and header
%----------------------------------------------------------------------------
% Environment definitions
\newenvironment*{mytitle}{\begin{LARGE}\bf}{\end{LARGE}\\}%
\newenvironment*{mysubtitle}{\bf}{\\[1.5ex]}%
\newenvironment*{myabstract}{\begin{Large}\bf}{\end{Large}\\[2.5ex]}%
%-----------------------------------------------------------------------------
% Using Pictures and tables:
% - Instead "table" write "tablehere" without parameters
% - Instead "figure" write "figurehere " without parameters
% - Please insert a blank line before and after \begin{figuerhere} ... \end{figurehere}
%
% CAUTION:   The first reference to a figure/table in the text should be formatted fat.
%
\makeatletter
\newenvironment{tablehere}{\def\@captype{table}}{}
\newenvironment{figurehere}{\def\@captype{figure}\vspace{2ex}}{\vspace{2ex}}
\makeatother



%%%%%%%%%%%%%%%%%%%%%%%%%%%%%%%%%%%%%%%%%%%%%%%%%%%%%%%%%%%%%%%%%%%%%%%%%%%%%%
\begin{document}

% Please use capital letters in the beginning of important words as for example
\begin{mytitle}Heterogeneous Scheduler\end{mytitle}
\begin{mysubtitle}Scheduling support for heterogeneous hardware accelerators under linux\end{mysubtitle}
%
% Please do not insert a line here
%
\\
Mambretti Andrea\\
Matr. 783286, (m4mbr3@gmail.com)\\
\begin{flushright}
\emph{Report for the master course of real time operative system}\\
\emph{Reviser: PhD. Patrick Bellasi (bellasi@elet.polimi.it)}
\end{flushright}

Received: March, 06 2013\\
\hspace{10ex}

\begin{myabstract} Abstract \end{myabstract}
During these days, computers are more often provided with dedicated graphic cards. Modern GPUs are
designed to exploit an high level of parallelism and compared to a standard CPU are faster
(sometimes more than 10X) and more expensive. 
A common user, that is not either playing or doing some graphic task, use the ~10\% of GPUs 
computational capacity. The main goal of this project is to fix, deploy and test an heterogeneous 
scheduler that allows linux to use both CPUs and GPUs at the same time. The implementation of this 
scheduler is in both kernel and user space and allow the tasks migration using the cuda libraries.


\vspace{4ex}	% Please do not remove or reduce this space here.
\begin{multicols}{2}

%%%%%%%%%%%%%%%%%%%%%%%%%%%%%%%%%%%%%%%%%%%%%%%%%%%%%%%%%%%%%%%%%%%%%%%%%%%%%
\begin{myabstract} Introduction \end{myabstract}
Since the producers of GPUs has released the first version of API to program directly those 
components, researchers and industries has started to produce hybrid solution of scheduler. 
The main goal of these solutions is the performance improvement. What is analyzed in this project is 
a system that use the CUDA API provided by NVIDIA and the linux kernel to create a huge scheduler
being able to migrate tasks between GPUs/CPUs and viceversa. This report is organized in such a way
that the first part provides an high level description of the architecture concepts. The 
second describes how this concepts has been implemented and tested. The third figures out step by step
the installation phases. 




\section{Architercture concepts}

%-----------------------------------------------------------------------------
\subsection{Heterogeneous system definition and problems}
% Please avoid separations in titles
% and separate text manually
The scheduling concepts presented in this project are based on heterogeneous systems that are 
those systems where there are non-uniform  computing unit which include single- or multi-core CPUs
operating in SMP mode and an arbitrary combination of additional hardware accelerators such as GPUs,
DSPs or FPGA.
Scheduling such systems is more difficult due to several reasons:\\
1) In general accelerator doesn't have shared memory space with CPU cores so every time two processes 
that are running on different computational unit, one on CPU and one on an accelerator, has to share
some piece of data they are obligated to trasfer the information. The scheduler in this case has to 
consider the transfer time between them. These overheads have to be known and used as input for a
scheduling decision in heterogeneous systems. Futhermore this problem introduces a course granularity 
in the scheduling due to the non-uniformity of communication bandwith between components, latency and
performance characteristics.\\
2) Most accelerator architectures do not support preeption but assume a run-to-completion execution
model. While computations on CPU cores can be easily preempted and resumed by reading and restoring
well defined internal registers, most hardware accelerators do no even expose teh complete internal
state nor are they designed to be interrupted.\\
3) Heterogeneous computing resources have completely different architectures and ISAs. Hence, 
a dedicated binary is required for each combination of task and accelerator which prevents migrating 
tasks between arbitrary compute units. Even if a task with the same functionality is avaiable for
several architectures and if the internal state of the architecture is accessible, misgrating a task
between different architectures is far from trivial, because the represetnation and interpretation of 
state is completely different.

\subsection{Design decisions}
During the implementation phase some decision have been taken to design the framework. \\






%-----------------------------------------------------------------------------
\subsection{The second subsection of the first \\ Section}

Lorem ipsum dolor sit amet, consectetur adipiscing elit. Donec et ligula. Nullam
in libero. Donec dictum pede in justo. Lorem ipsum dolor sit amet, consectetur
adipiscing elit. Aliquam congue. Aliquam egestas. Nunc eu est ac nibh mattis
vestibulum. Curabitur aliquet bibendum odio. Etiam hendrerit. Nunc a velit quis
dui molestie consequat. Sed et turpis et mi feugiat tincidunt. Sed sollicitudin.
Ut risus? Duis eget orci eu turpis consectetur fringilla? Lorem ipsum dolor sit
amet, consectetur adipiscing elit. Nullam tellus ligula, placerat vitae, tempor
vitae, varius id; est! Nullam et ipsum eget tellus eleifend sollicitudin? Fusce
urna massa, imperdiet vitae, convallis in; lacinia sed, tortor.



%%%%%%%%%%%%%%%%%%%%%%%%%%%%%%%%%%%%%%%%%%%%%%%%%%%%%%%%%%%%%%%%%%%%%%%%%%%%%
\section{The Second Section}

Lorem ipsum dolor sit amet, consectetur adipiscing elit.  Aenean magna. Nunc non
ante eget nibh condimentum tempor. Nullam ullamcorper lectus eget mauris. Nam
neque orci; rhoncus at, pulvinar quis, elementum sit amet, turpis. Mauris
posuere nisi ut justo. Morbi non lorem vitae mauris interdum faucibus.
Vestibulum ut sapien in augue faucibus fringilla. Vestibulum ante ipsum primis
in faucibus orci luctus et ultrices posuere cubilia Curae; Etiam vestibulum
fringilla libero. Curabitur libero diam, hendrerit sit amet, ornare eget,
imperdiet vel, purus!


%-----------------------------------------------------------------------------
\subsection{The first subsection of the second \\ Section}

Lorem ipsum dolor sit amet, consectetur adipiscing elit. Nam consectetur ante at
eros. Vestibulum mi nisi, venenatis sollicitudin, tempus sed, auctor id, tortor.
Fusce orci. Duis tellus arcu, euismod sed, consequat sit amet, elementum vel,
mauris. Curabitur leo diam; dapibus quis, condimentum vitae, dignissim ut, diam.
Nulla et nulla eget elit volutpat sagittis.

%-----------------------------------------------------------------------------
\subsection{The second subsection of the second \\ Section}

Lorem ipsum dolor sit amet, consectetur adipiscing elit. Mauris eget mauris.
Nulla facilisi. Ut condimentum tempor eros? Integer metus mauris, consectetur
sit amet, tempor a, facilisis eu, nisl. Vestibulum at turpis. Ut vitae tortor
pretium nisl vestibulum blandit. Nulla nibh urna, semper et, elementum at,
mattis ut, nisi! Cum sociis natoque penatibus et magnis dis parturient montes,
nascetur ridiculus mus. Morbi vel ligula eget lacus convallis venenatis. Aliquam
lacinia tincidunt felis. Ut dui.

% We suggest the use of JabRef for editing your bibliography file (Report.bib)
%\bibliographystyle{splncs}
%\bibliography{Report}

\end{multicols}
\end{document}
