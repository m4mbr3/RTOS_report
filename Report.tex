% VDE Template for EUSAR Papers
% Provided by Barbara Lang und Siegmar Lampe
% University of Bremen, January 2002
% English version by Jens Fischer
% German Aerospace Center (DLR), December 2005
% Additional modifications by Matthias Wei{\ss}
% FGAN, January 2009

%-----------------------------------------------------------------------------
% Type of publication
\documentclass[a4paper,10pt]{article}
%-----------------------------------------------------------------------------
% Other packets: Most packets may be downloaded from www.dante.de and
% "tcilatex.tex" can be found at (December 2005):
% http://www.mackichan.com/techtalk/v30/UsingFloat.htm
% Not all packets are necessarily needed:
\usepackage[T1]{fontenc}
\usepackage[latin1]{inputenc}
%\usepackage{ngerman} % in german language if required
\usepackage[nooneline,bf]{caption} % Figure descriptions from left margin
\usepackage{times}
\usepackage{multicol}
\usepackage{amsmath}
\usepackage{amssymb}
\usepackage[dvips]{graphicx}
\usepackage{epsfig}
\input{tcilatex}
%-----------------------------------------------------------------------------
% Page Setup
\textheight24cm \textwidth17cm \columnsep6mm
\oddsidemargin-5mm                 % depending on print drivers!
\evensidemargin-5mm                % required margin size: 2cm
\headheight0cm \headsep0cm \topmargin0cm \parindent0cm
\pagestyle{empty}                  % delete footer and header
%----------------------------------------------------------------------------
% Environment definitions
\newenvironment*{mytitle}{\begin{LARGE}\bf}{\end{LARGE}\\}%
\newenvironment*{mysubtitle}{\bf}{\\[1.5ex]}%
\newenvironment*{myabstract}{\begin{Large}\bf}{\end{Large}\\[2.5ex]}%
%-----------------------------------------------------------------------------
% Using Pictures and tables:
% - Instead "table" write "tablehere" without parameters
% - Instead "figure" write "figurehere " without parameters
% - Please insert a blank line before and after \begin{figuerhere} ... \end{figurehere}
%
% CAUTION:   The first reference to a figure/table in the text should be formatted fat.
%
\makeatletter
\newenvironment{tablehere}{\def\@captype{table}}{}
\newenvironment{figurehere}{\def\@captype{figure}\vspace{2ex}}{\vspace{2ex}}
\makeatother



%%%%%%%%%%%%%%%%%%%%%%%%%%%%%%%%%%%%%%%%%%%%%%%%%%%%%%%%%%%%%%%%%%%%%%%%%%%%%%
\begin{document}

% Please use capital letters in the beginning of important words as for example
\begin{mytitle}Heterogeneous Scheduler\end{mytitle}
\begin{mysubtitle}Scheduling support for heterogeneous hardware accelerators under linux\end{mysubtitle}
%
% Please do not insert a line here
%
\\
Mambretti Andrea\\
Matr. 783286, (m4mbr3@gmail.com)\\
\begin{flushright}
\emph{Report for the master course of real time operative system}\\
\emph{Reviser: PhD. Patrick Bellasi (bellasi@elet.polimi.it)}
\end{flushright}

Received: March, 06 2013\\
\hspace{10ex}

\begin{myabstract} Abstract \end{myabstract}
During these days, computers are more often provided with dedicated graphic cards. Modern GPUs are
designed to exploit an high level of parallelism and compared to a standard CPU are faster
(sometimes more than 10X) and more expensive. 
A common user, that is not either playing or doing some graphic task, use the ~10\% of GPUs 
computational capacity. The main goal of this project is to fix, deploy and test an heterogeneous 
scheduler that allows linux to use both CPUs and GPUs at the same time. The implementation of this 
scheduler is in both kernel and user space and allow the tasks migration using the cuda libraries.


\vspace{4ex}	% Please do not remove or reduce this space here.
\begin{multicols}{2}

%%%%%%%%%%%%%%%%%%%%%%%%%%%%%%%%%%%%%%%%%%%%%%%%%%%%%%%%%%%%%%%%%%%%%%%%%%%%%
\begin{myabstract} Introduction \end{myabstract}
Since the producers of GPUs has released the first version of API to program directly those 
components, researchers and industries has started to produce hybrid solution of scheduler. 
The main goal of these solutions is the performance improvement. What is analyzed in this project is 
a system that use the CUDA API provided by NVIDIA and the linux kernel to create a huge scheduler
being able to migrate tasks between GPUs/CPUs and viceversa. This report is organized in such a way
that the first part provides an high level description of the architecture concepts. The 
second describes how this concepts has been implemented and tested. The third figures out step by step
the installation phases. 




\section{Architercture concepts}

%-----------------------------------------------------------------------------
\subsection{Heterogeneous system definition and problems}
% Please avoid separations in titles
% and separate text manually
The scheduling concepts presented in this project are based on heterogeneous systems that are 
those systems where there are non-uniform  computing unit which include single- or multi-core CPUs
operating in SMP mode and an arbitrary combination of additional hardware accelerators such as GPUs,
DSPs or FPGA.
Scheduling such systems is more difficult due to several reasons:\\
1) In general accelerator doesn't have shared memory space with CPU cores so every time two processes 
that are running on different computational unit, one on CPU and one on an accelerator, has to share
some piece of data they are obligated to trasfer the information. The scheduler in this case has to 
consider the transfer time between them. These overheads have to be known and used as input for a
scheduling decision in heterogeneous systems. Futhermore this problem introduces a course granularity 
in the scheduling due to the non-uniformity of communication bandwith between components, latency and
performance characteristics.\\
2) Most accelerator architectures do not support preeption but assume a run-to-completion execution
model. While computations on CPU cores can be easily preempted and resumed by reading and restoring
well defined internal registers, most hardware accelerators do no even expose teh complete internal
state nor are they designed to be interrupted.\\
3) Heterogeneous computing resources have completely different architectures and ISAs. Hence, 
a dedicated binary is required for each combination of task and accelerator which prevents migrating 
tasks between arbitrary compute units. Even if a task with the same functionality is avaiable for
several architectures and if the internal state of the architecture is accessible, misgrating a task
between different architectures is far from trivial, because the represetnation and interpretation of 
state is completely different.

\subsection{Design decisions}
During the implementation phase some decision have been taken to design the framework. \\

The first was about the Scheduler Component: Scheduling of homogeneous CPU cores is currently done in
the kernel, as all needed input information for the scheduling decision is avaiable to the system, 
so that the scheduling problem can be completely hidden from the application programmer. The 
heterogeneous scheduling problem is more complicated, as more decision parameters affect the decision,
which are partly not avaiable to the systems scheduler component.\\
Selecting an appropriate hardware architecture for a task to be sceduled dynamically at runtime
is not trivial and has to be performed  by a scheduler, which can be located at different locantions
in the system, either in the application, in user space or in the system's kernel.\\
To allow a holistic view on the applications and its execution environment, we perform scheduling in 
the system's kernel by extending the CF scheduler. That way the scheduling principles are still 
hidden from the application developer and the OS can perform global decisions based on the system
utilization. Application specific scheduling inputs still have to be provided by the application 
developer to incorporate application's needs. Therefore we use a hybrid user/kernel level approach
to perform heterogeneous scheduling. A specific interface has to be provided to allow communication
between  application and scheduler.

The second was about the Operating System Adapting: Kernel space scheduling is the current standard
in operating systems. To provide support for heterogeneous architectures one could either extend an 
existing OS or completely rewrite and fully optimize it towards the heterogeneity. While heterogenous
systems will be more and more used in future and become standard in a foreseeable time, the authors
believe that a complete rewrite of OS is not needed. An extension to the current  system has several
advantages: Providing a modular implemented extension to the CFS 1) keeps teh management structures
as well as the scheduler component exchangeble, 2) makes the changes easily applicable to other OS,
and 3) reuses well established and well known functionalities of the current  kernel that have been
developed over years. That way our kernel extension will help to explore new directions for future OS,
but does not yet try to set a new standard.

The third decision was about Delegate Threads: Tasks that execute on heterogeneous resources may have
no access to main memory and use a completely different instruction set or execution model than an
equivalent task on a CPU. In order to schedule and manage these tasks whitout requiring a major OS
rewrite, we need to expose the tasks to the OS as known schedulable entities. We therefore represent
each task executing on a hardware accelerator  as a thread  to the OS. This allows us to use and 
extend  the existing data structures of the scheduler  in the linux kernel. We denote each thread 
representing a task on a hardware accelerators as delegate thread.\\
Apart from serving as a schedulable entity, the delegate thread also performs all operating system
interaction and managment operations on behalf of the task executing on the accelerator unit, such as
trasferring  data to and from the compute unit and controlling its configuration and execution.
The delegate threads must be spawned explicitly by the application and thus can also be used for
co-scheduling on different architectures. Once created, all threads are treated and scheduled equally
by the operating system.

As a forth decision there is the Cooperative Multitasking: The CFS implements preemptive multitasking
with time-sharing based on a fairness measure. Therefore, our scheduler has to include means to 
preempt a task and to migrate it to another computing unit. While non-voluntary preemption on FPGAs is
possible, GPUs currently do not directly support it yet, even if it is planned for the future.
Therefore we use the delegate threads to forward requests from the kernel scheduler to the task on 
the accelerator.
Nevertheless, even enabling preemption on GPUs does not solve the migration problem. The major 
difficulty is to find a way of mapping  the current  state of a compute unit to an equivalent state
on a different compute unit. To allow preemption and subsequent migration of applications on 
heterogeneous systems, their delegate threads need to be in a state, which can be interpreted by other
accelerators or by the CPU. As it is no possible to interrupt an accelerator at an arbitrary point
of time and to assume that it is in such a state, we propose to use a cooperative multitasking 
approach using checkpoints  to resolve these limitations. After reaching a checkpoint, an application
voluntarily hands back the control to the OS, which then may perform scheduling decisions to suspend 
and migrate a thread at these points. The authors beleive that this currently is the only way to 
simulate preemptive  multitasking on heterogeneous hardware.

\subsection{Scheduling Model}

From the design decisions above we derive our scheduling model that is not restricted to a certain 
class of operating systems or scheduling algorithms. Applications using the scheduler may spawn 
several threads that may possibly run on diverse architectures.

Thread information: As the scheduler needs information about the threads to be scheduled, the authors
store this application provided information called meta information about each thread and submit it 
to the scheduler. The meta information can be individually set for an application. Currently the 
authors only use a type affinity towards a targe architecture which can be determined dynamically
depending on  the input data. Further application specific input data can possibly be determined 
using profiling prior to the first use of an application.

Scheduling: The scheduler component may be located in the kernel space as well as the user space. To
assign tasks to certain hardware components, the scheduler has to provide a queue for each available
hardware. The application provided meta information is used in a scheduling policy to map newly 
arriving tasks to one of the queues. Whenever a compute unit runs indle or the currently running 
task has used  its complete time slice, the scheduler may dequeue a waiting task for that specific
compute unit. In case this is a hardware task, the delegate thread receives the information that it
may run its hardware counterpart. This includes using   the proprietary drivers of the hardware, which
are inevitable for the communication with some accelerators. As these currently may only be used from
used space, this requires a combined approach using the kernel space and the user space. For CPUs, 
the standard Linux scheduler is used.

Checkpointing: Checkpointing has to be performed when the application can safely interrupt its 
execution and store its state in main memory. The state has to be stored by the application itself
in data structures of the corresponding delegate thread, which then can be migrated to a different
architecture. The checkpoint data of the delegate thread thus has to be readable by all target 
architectures.
We define a checkpoint as a struct of data structures that unambiguosisly defines the state of the 
application. The scheduler does not have any requirements concerning the checkpoint data. Hence, the
application has to make sure that all needed data is available in these data structures and thus
stored in accessible memory at the end of each thread's time slice. A checkpoint in most cases is a
combination of 1) a set of data structures that define a minimum state that is reached several times
during execution, and 2) a data structure that define the position in the code. The checkpoint data
of an application is copied to the newly allocated accelerator and copied back to the host's main 
memory when the application's time slice is exhausted. 
Checkpoints are to be defined by the application developer or to be inserted by a compiler. One has 
to identify a preferably small set of data structures that 1) unambiguously define the state of a 
thread, and 2) are readable and translatable to corresponding data structures of other compute units.
The size of checkpoints may vary to a large extend depending on the application used. While MD5 
cracking  only needs to store the current loop index and the given search-string, image processing 
algoritms require to store the complete intgrermediate results that might be of large extent.
In general, a checkpoint could be simply defined by a list of already processed data sets. Therefore,
the choice of the checkpoint is very important and influences the scheduling granularity. The 
checkpoint distance, i.e., the amount of work done between 2 checkpoints stored back, increases with
the size of the checkpoint.
We here assume all checkpoints to be small enough to fit into the host's memory. The introduced 
checkpoint size is known at definition time and may be used to re-determine the scheduling granularity
for a task.







%-----------------------------------------------------------------------------
\subsection{The second subsection of the first \\ Section}

Lorem ipsum dolor sit amet, consectetur adipiscing elit. Donec et ligula. Nullam
in libero. Donec dictum pede in justo. Lorem ipsum dolor sit amet, consectetur
adipiscing elit. Aliquam congue. Aliquam egestas. Nunc eu est ac nibh mattis
vestibulum. Curabitur aliquet bibendum odio. Etiam hendrerit. Nunc a velit quis
dui molestie consequat. Sed et turpis et mi feugiat tincidunt. Sed sollicitudin.
Ut risus? Duis eget orci eu turpis consectetur fringilla? Lorem ipsum dolor sit
amet, consectetur adipiscing elit. Nullam tellus ligula, placerat vitae, tempor
vitae, varius id; est! Nullam et ipsum eget tellus eleifend sollicitudin? Fusce
urna massa, imperdiet vitae, convallis in; lacinia sed, tortor.



%%%%%%%%%%%%%%%%%%%%%%%%%%%%%%%%%%%%%%%%%%%%%%%%%%%%%%%%%%%%%%%%%%%%%%%%%%%%%
\section{The Second Section}

Lorem ipsum dolor sit amet, consectetur adipiscing elit.  Aenean magna. Nunc non
ante eget nibh condimentum tempor. Nullam ullamcorper lectus eget mauris. Nam
neque orci; rhoncus at, pulvinar quis, elementum sit amet, turpis. Mauris
posuere nisi ut justo. Morbi non lorem vitae mauris interdum faucibus.
Vestibulum ut sapien in augue faucibus fringilla. Vestibulum ante ipsum primis
in faucibus orci luctus et ultrices posuere cubilia Curae; Etiam vestibulum
fringilla libero. Curabitur libero diam, hendrerit sit amet, ornare eget,
imperdiet vel, purus!


%-----------------------------------------------------------------------------
\subsection{The first subsection of the second \\ Section}

Lorem ipsum dolor sit amet, consectetur adipiscing elit. Nam consectetur ante at
eros. Vestibulum mi nisi, venenatis sollicitudin, tempus sed, auctor id, tortor.
Fusce orci. Duis tellus arcu, euismod sed, consequat sit amet, elementum vel,
mauris. Curabitur leo diam; dapibus quis, condimentum vitae, dignissim ut, diam.
Nulla et nulla eget elit volutpat sagittis.

%-----------------------------------------------------------------------------
\subsection{The second subsection of the second \\ Section}

Lorem ipsum dolor sit amet, consectetur adipiscing elit. Mauris eget mauris.
Nulla facilisi. Ut condimentum tempor eros? Integer metus mauris, consectetur
sit amet, tempor a, facilisis eu, nisl. Vestibulum at turpis. Ut vitae tortor
pretium nisl vestibulum blandit. Nulla nibh urna, semper et, elementum at,
mattis ut, nisi! Cum sociis natoque penatibus et magnis dis parturient montes,
nascetur ridiculus mus. Morbi vel ligula eget lacus convallis venenatis. Aliquam
lacinia tincidunt felis. Ut dui.

% We suggest the use of JabRef for editing your bibliography file (Report.bib)
%\bibliographystyle{splncs}
%\bibliography{Report}

\end{multicols}
\end{document}
